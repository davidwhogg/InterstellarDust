% This document is part of the Dust project.
% Copyright 2014 David W. Hogg (NYU)

\documentclass[12pt, letterpaper]{article}

% math, bitches
\newcommand{\dd}{\mathrm{d}}

\begin{document}

\section{priors}

Imagine that we are trying to infer the dust density $\rho(x)$, where
$\rho$ is a continuous function of three-dimensional position $x$.
We will ignore velocity space for now!
Imagine that our prior belief is that the dust density $\rho(x)$ is a
stochastic function that can be generated (approximately) by a
Gaussian with some sensible mean function $\mu(x)$ and some variance
tensor $V(\Delta x)$.
The variance tensor acts on differences $\Delta x$ of position; more
on this later.
The fact that $\rho(x)$ is infinite dimensional is a detail; if it
makes you upset, imagine that $x$ can only take on values on a very
fine, finite grid.

This prior belief---that the dust density is drawn from a
Gaussian---is sensible, because it is easily manipulated and computed.
That said, this prior belief is insane, because it allows for
\emph{negative} dust.
There is no way to remove the support of the Gaussian on the
non-fully-positive part of the space (and this is the vast majority of
the space).
We could go to a Gaussian in the \emph{logarithm} of the density, but
this will break the linearity we are going to use (below).

Now think about the fact that we only ever observe the extinction
$A(x)$ at some sparse set of three-dimensional positions $x$; we never
directly observe the density $\rho(x)$.
Following the Bailer-Jones notation, these are related to density by
\begin{eqnarray}
A(x) &=& \kappa\,\int_0^1 \rho(\eta\,x)\,|x|\,\dd\eta
\quad,
\end{eqnarray}
where $\eta$ is a dimensionless integration variable.

If it is our prior belief about $\rho(x)$ that it is drawn from a
Gaussian with mean $\mu(x)$ and variance tensor $V(\Delta x)$ then I
\emph{believe} that the consistent prior belief about $A(x)$ is that
it is drawn from a Gaussian with mean $\mu'(x)$ and variance tensor
$V'(\Delta x)$ with
\begin{eqnarray}
\mu'(x) &=& \kappa\,\int_0^1 \mu(\eta\,x)\,|x|\,\dd\eta
\\
V'(x_1 - x_2) &=& \kappa^2\,\int_0^1\int_0^1 V(\eta_1\,x_1 - \eta_2\,x_2)\,|x_1|\,|x_2|\,\dd\eta_1\,\dd\eta_2
\quad .
\end{eqnarray}
That is, a Gaussian process in $\rho(x)$ implies a Gaussian process in
$A(x)$.

\section{likelihoods}

Now, how does inference of $A(x)$ and $\rho(x)$ proceed?

\section{posteriors}

\section{kernel and hyper-parameters}

\section{open issues}

\end{document}
