\documentclass[12pt, letterpaper]{article}

\newcommand{\given}{\,|\,}

\begin{document}

Imagine you have $M$ two-dimensional pixels $m$ on the celestial sphere.
Within each pixel there are $N$ three three-dimensional boxels $mn$ in space along the line of sight
  (such that there are $[M\,N]$ total boxels in space).
Each boxel $mn$ has position $r_{mn}$ and we imagine it contains mean dust mass density $\rho_{mn}$.
We don't know these densities;
  they are parameters from our perspective.
For notational convenience we will define two other kinds of density objects:
One is a set of $M$ $N$-vectors $\rho_m$, one per sky pixel $m$,
  each of which is the vector made up of the $N$ values $\rho_{mn}$ in sky pixel $m$.
Another is the full $[M\,N]$-vector $\rho$,
  which is the single complete vector of all boxel densities $\rho_{mn}$.

Eddie and company have chosen some simple, vague prior on the $\rho_{mn}$
  (something like flat in $\rho_{mn}$ or $\ln\rho_{mn}$, and something like independent from boxel to boxel).
We will call this prior the ``interim prior'' in what follows.
For what follows, all we need to know are the following two things:
First, given the full $[M\,N]$-vector $\rho$,
  we can evaluate the interim prior $p_0(\rho)$.
Indeed, since Eddie and co.\ have sampled each sky pixel independently,
  this full-space prior is a product of individual-pixel priors.
Second, for each sky pixel $m$, we have a $K$-element sampling $\rho_{m}^{(k)}$,
  each of which is a $N$-vector $\rho_m$ drawn from the posterior pdf
  for the density along the line of sight in pixel $m$.

Now, one beautiful thing we will use is that,
  because each boxel $m$ has been independently sampled,
  and has independent data,
  and has an independent prior pdf over $\rho_m$,
  any combinatoric combination of the samplings is itself a sampling.
That is, the $K$ samplings per pixel $m$
  leads to a total of $K^M$ valid samplings for the full density $[M\,N]$-vector $\rho$.
Because the density is inferred from stars,
  because stars constrain integrals of the density,
  and because stars have noisy distance measurements,
  we only have noisy, correlated, non-trivial beliefs about the boxels $\rho_{mn}$ within sky pixel $m$,
  conditioned on this interim prior.

Imagine we want to---after the sampling under the interim prior has completed---%
  impose some more informative spatial prior pdf $p(\rho\given\alpha)$
  on the big density parameter vector $\rho$.
(Here $\alpha$ is a vector or blob of hyperparameters that control the informative spatial prior.)
Can we do this without going back to the original sampling?
Yes, we can, although it might hurt a bit.
It involves importance sampling.
The idea is as follows:
If we obtain a posterior sample $\rho^{(j)}$ of $\rho$ drawn from the posterior
  created with the interim prior $p_0(\rho)$,
  we can compute a weight $w_j$ for this sample
\begin{eqnarray}
w_j & \leftarrow & \frac{p(\rho^{(j)}\given\alpha)}{p_0(\rho^{(j)})}
\quad,
\end{eqnarray}
  where the weights are just the ratio of new prior to old prior (informative to interim).
The \emph{weighted} set of samples is, in principle,
  a sampling from the posterior pdf constructed under the informative prior.
Problem solved!

The problems are substantial:
The vast majority of samples we have of $\rho$ will have extremely low weights in this calculation.
There \emph{is} a vast number of samples of $\rho$ (combinatoric; see above).
It is very hard to find the small fraction (vanishing fraction) of samples that are ``important''
  under the importance sampling.

The \textbf{dumbest possible option} is as follows:


\end{document}
